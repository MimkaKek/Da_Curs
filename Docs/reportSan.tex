\documentclass[12pt]{article}
\usepackage{indentfirst}
\usepackage{fullpage}
\usepackage{multicol,multirow}
\usepackage{tabularx}
\usepackage{ulem}
\usepackage[utf8]{inputenc}
\usepackage[russian]{babel}
\usepackage{listings}

\begin{document}
\begin{titlepage}
	\begin{center}
		\bfseries
		
		{\Large Московский авиационный институт\\ (национальный исследовательский университет)
			
		}
		
		\vspace{48pt}
		
		{\large Факультет информационных технологий и прикладной математики
		}
		
		\vspace{36pt}
		
		
		{\large Кафедра вычислительной математики и~программирования
			
		}
		
		
		\vspace{48pt}
		
		Курсовая работа по курсу <<Дискрeтный анализ>>: Методы сжатия данных
	\end{center}
	
	\vspace{72pt}
	
	\begin{flushright}
		\begin{tabular}{rl}
			Студент: & А.\,М. Титеев  \\
			Преподаватель: & Н.\,А. Зацепин \\
			Группа: & М8О-408Б \\
			Дата: & \\
			Оценка: & \\
			Подпись: & \\
		\end{tabular}
	\end{flushright}
	
	\vfill
	
	\begin{center}
		\bfseries
		Москва, \the\year
	\end{center}
\end{titlepage}

\pagebreak
\subsection*{Условие}

Необходимо реализовать два известных метода сжатия данных для сжатия одного файла. 

Формат запуска должен быть аналогичен формату запуска программы gzip, должны быть поддержаны следующие ключи: c, d, k, l, r, t, 1, 9. Должно поддерживаться указание символа дефиса в качестве стандартного ввода.

\subsection*{Метод решения}

Как и требуется в условии запуск программы аналогичен запуску утилиты gzip: ./main <ключи> <файлы> <ключи> <файлы> ...

\subsubsection*{Обработка входных данных}

На первом этапе работы программа определяет наличие в поступившей строке ключей, директорий и файлов.
При обработке ключей учитывается их взаимоперекрываение, как в утилите gzip: l и r имеют наибольший приоритет, далее идёт ключ t, после чего остальные. В случае, если новый ключ перекрывает по логике утилиты некоторые из уже имеющихся, то эти ключи деактивируются.

%TODO У саняька чтение всех файлов из папки до работы с файлами

Если полученное слово из стандартного ввода не является ключом, то программа проверяет наличие директории с таким именем. Если такой директории нет, то считается, что это имя файла, и оно заносится в список файлов. Если директория с таким именем существует и подключён ключ r, то все файлы внутри этой директории добавляются в список.

После с файлами ведётся работа согласно введённым ключам.

\subsubsection*{Арифметическое кодирование}%TODO Слава
. 

\subsubsection*{LZ77}%TODO Саня

%TODO куда записать принцип рабты с файлами?

\subsection*{Описание файлов программы}

Код программы разбит на 11 файлов:%TODO переименовать и перетасовать файлы

\begin{enumerate}
	\item Arithmetic.h - Содержит перечисление методов и описание класса TArithmetic, необходимого для работы арифметической компрессии и декомпрессии. 
	\item Arithmetic.cpp - Содержит реализацию всех методов класса TArithmetic.
	\item BFile.h - Содержит перечисление методов и описание класса TOutBinary и класса TInBinary, необходимых для записи в файл и чтения из файла соответственно.
	\item BFile.cpp - Содержит реализацию всех методов классов TOutBinary и TInBinary.
	\item Globals.h - Содержит в себе все необходимые глобальные переменные и библиотеки используемые несколькими файлами.
	\item LZ77.h - Содержит перечисление методов и описание класса TLZ77, необходимого для работы алгоритма LZ77.
	\item LZ77.cpp - Содержит реализацию всех методов класса TLZ77.
	\item main\_help.h - Содержит в себе перечисление и описание всех функций необходимых для препроцессинга перед началом работы алгоритмов компрессии и декомпрессии.
	\item main\_help.cpp - Содержит реализацию всех функций, необходимых для препроцессинга, описанных в файле main\_help.h.
	\item main.cpp - Содержит в себе алгоритм чтения файлов и ключей.
	\item Makefile - Файл для сборки программы.
\end{enumerate}

\subsection*{Основные типы данных}

\begin{enumerate}%TODO
	\item TArithmetic - класс, описывающий работу арифметического алгоритма компрессии и декомпрессии.
	\item TOutBinary - класс обеспечивающий запись необходимого количества байт в файл.
	\item TInBinary - класс обеспечивающий считывание необходимого количества байт из файла.
	\item TLZ77 - класс, описывающий работу алгоритма LZ77.
	\item 
\end{enumerate}

\subsection*{Описание методов и функций программы}
 
\subsubsection*{Основные свойства и методы класса TArithmetic}%TODO Слава внимательно посмотрите как я это оформил в своих subsubsection
\noindent
public:

\begin{enumerate}
	\item 
	\item 
	\item 
	\item 
	\item 
	\item 
	\item 
	\item 
\end{enumerate}
\noindent
private:

\begin{enumerate}
	\item 
	\item 
	\item 
	\item 
	\item 
	\item 
	\item 
	\item 
\end{enumerate}

\subsubsection*{Основные свойства и методы класса TOutBinary}%TODO Слава
\noindent
public:

\begin{enumerate}
	\item 
	\item 
	\item 
	\item 
	\item 
	\item 
	\item 
	\item 
\end{enumerate}
\noindent
private:

\begin{enumerate}
	\item 
	\item 
	\item 
	\item 
	\item 
	\item 
	\item 
	\item 
\end{enumerate}

\subsubsection*{Основные свойства и методы класса TInBinary}%TODO Слава
\noindent
public:

\begin{enumerate}
	\item 
	\item 
	\item 
	\item 
	\item 
	\item 
	\item 
	\item 
\end{enumerate}
\noindent
private:

\begin{enumerate}
	\item 
	\item 
	\item 
	\item 
	\item 
	\item 
	\item 
	\item 
\end{enumerate}

\subsubsection*{Основные свойства и методы класса TLZ77}%TODO Саня
\noindent
public:

\begin{enumerate}
	\item 
	\item 
	\item 
	\item 
	\item 
	\item 
	\item 
	\item 
\end{enumerate}
\noindent
private:

\begin{enumerate}
	\item 
	\item 
	\item 
	\item 
	\item 
	\item 
	\item 
	\item 
\end{enumerate}

\subsubsection*{Прочие функции} %TODO переименовать и перетасовать функции
\noindent
\begin{enumerate}
	\item bool KeyManager(std::string) - Обрабатывает полученные ключи. В случае получения неизвестного ключа возвращает false, иначе true.
	\item bool DifferensOfSizes(TInBinary*, std::string) - вывод для каждого файла размера сжатого, оригинального, коэффициента сжатия(\%) и имя оригинального файла(ключ l). В случае повреждения. архива возвращает false, иначе true.
	\item void WorkWithDirectory(std::string) - работает с директорией (ключ r).
	\item void WorkWithFile(std::string) - работает с файлом (определяет наличие файла, принимает решение о компрессии или декомпрессии, выполняет прочие ключи).
	\item bool IsDirectory(std::string, bool) - Проверяет, является ли файл директорией. Если файл является директорией, возвращает true, иначе false.
	\item void PrintDirectoryErrors(std::string) - Уведомляет об ошибках.
	\item bool IsArchive(std::string) - Проверяет, является ли файл архивом. Если файл является архивом, возвращает true, иначе false.
	\item void Rename(std::string, std::string) - Изменяет название файла после успешной компрессии или декомпрессии.
	\item void Delete(std::string) - Удаляет временный файл.
	\item void MainDecompress(TInBinary*, std::string) - Отвечает за подготовку декомпрессинга.
	\item void MainCompress(TInBinary*, std::string) - Отвечает за подготовку компрессинга.
	\item unsigned long long int LZWCompress(TInBinary*, std::string, TOutBinary*) - Подготавливает LZW компрессию. Возвращает размер нового файла.
	\item unsigned long long int LZ77Compress(TInBinary*, std::string, TOutBinary*) - Подготавливает LZ77 компрессию. Возвращает размер нового файла.
	\item unsigned long long int ArithmeticCompress(TInBinary*, std::string) - Подготавливает арифметический компрессию. Возвращает размер нового файла.
	\item void KeepSmall(unsigned long long int, unsigned long long int,
	unsigned long long int, std::string) - Сохраняет архив самого малого размера.
	\item int main(int, char*) - Осуществляет чтение входных данных.
\end{enumerate}

\subsection*{Исходный код}%TODOпоходу его будем вставлять как всё закончим

\subsection*{Тест производительности}%TODO это нормально затестить получится токо когда всё будет

Для тестирования производительности использовалась английская версия книги Война и мир, размером 3.2 Мб.

\noindent
\begin{tabular}{| l | l | l | l | l | l |}
\hline
& Время & Время	& Сжатый & Пиковое & Коэффициент\\
& архивации & разъархивации	& размер & потребление & сжатия\\
& (с) & (с) & & памяти (Кб) &\\\hline
ключ 1 & & & & & \\\hline
ключ 9 & & & & & \\\hline
нет ключей & & & & & \\\hline
gzip & & & & & \\\hline
	
\end{tabular}

%TODO ВВеди характеристики компа

\subsection*{Выводы}%TODO



\subsection*{Список литературы}
\begin{enumerate}
	\item Арифметическое кодирование - Arithmetic coding [Электронный ресурс]:\\ ru.qwe.wiki URL: https://ru.qwe.wiki/wiki/Arithmetic\_coding (дата обращения\\ 28.08.2020)
	\item Arithmetic Coding [Электронный ресурс]: users.cs.cf.ac.uk URL:\\ https://users.cs.cf.ac.uk/Dave.Marshall/Multimedia/node213.html (дата обращения 16.09.2020)
	\item LZ77 на C, реализация алгоритма LZ77 на C [Электронный ресурс]:\\ algor.skyparadise.org URL: https://algor.skyparadise.org/read/14 (дата обращения 16.09.2020)
\end{enumerate}
\end{document}