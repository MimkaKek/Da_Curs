\documentclass[12pt]{article}
\usepackage{indentfirst}
\usepackage{fullpage}
\usepackage{multicol,multirow}
\usepackage{tabularx}
\usepackage{ulem}
\usepackage[utf8]{inputenc}
\usepackage[russian]{babel}
\usepackage{listings}
\usepackage{color}
\usepackage{graphicx}
\DeclareGraphicsExtensions{.png}

\begin{document}
	\begin{titlepage}
		\begin{center}
			\bfseries
			
			{\Large Московский авиационный институт\\ (национальный исследовательский университет)
				
			}
			
			\vspace{48pt}
			
			{\large Факультет информационных технологий и прикладной математики
			}
			
			\vspace{36pt}
			
			
			{\large Кафедра вычислительной математики и~программирования
				
			}
			
			
			\vspace{48pt}
			
			Курсовая работа по курсу <<Дискрeтный анализ>>: Методы сжатия данных
		\end{center}
		
		\vspace{72pt}
		
		\begin{flushright}
			\begin{tabular}{rl}
				Студент: & В.\,В. Гринин  \\
				Преподаватель: & Н.\,А. Зацепин \\
				Группа: & М8О-408Б \\
				Дата: & \\
				Оценка: & \\
				Подпись: & \\
			\end{tabular}
		\end{flushright}
		
		\vfill
		
		\begin{center}
			\bfseries
			Москва, \the\year
		\end{center}
	\end{titlepage}
	
	\pagebreak
	
	\subsection*{Условие}
	
	Необходимо реализовать два известных метода сжатия данных: LZW и Арифметическое кодирование. 
	
	Формат запуска должен быть аналогичен формату запуска программы gzip, должны быть поддержаны следующие ключи: -c, -d, -k, -l, -r, -t, -1, -9. Должно поддерживаться указание символа дефиса в качестве стандартного ввода.
	
	\subsection*{Метод решения}
	
	\subsubsection*{Обработка входных данных}
	
	Для начала программа обрабатывает аргументы командной строки. Таковыми могут быть ключи или наименования файлов и директорий. Чтобы отделять ключи от обычных наименований, в качестве первого символа для ключей используется <<->>. В ходе изучения работы ключей программы gzip, были выявлены закономерности в их поведении. А именно:
	
	\begin{enumerate}
		\item Если был введён ключ -t, то он делает ключи -c, -k, -d, -1, -9 недействительными как сейчас, так и при их будущих вводах.
		\item Если был введён ключ -l, то он делает ключи -c, -k, -d, -t, -1, -9 недействительными как сейчас, так и при их будущих вводах.
		\item В то же время сочетания -cd, -c1, -c9, -kd, -k1, -k9, а так же сочетания ключа -r со всеми остальными ключами не являются взаимоисключающими.
		\item Если же во время обработки ключей встречается неизвестный ключ, то программа прекращает свою работу с соответствующей ошибкой, как и утилита gzip.
	\end{enumerate}
	
	В случае если начальным символом не является <<->>, то оно заносится в красно-чёрное дерево имён файлов/директорий. 
	
	\subsubsection*{Работа с файлами}
	
	Проверяется пустота дерева имён файлов/папок -- если дерево пустое то программа завершается. 
	
	Если дерево не пустое, каждый его элемент сначала рассматривается как директория, затем как файл. Чтобы отделить директорию от файла, используется функция \texttt{opendir} из библиотеки \texttt{dirent.h}.
	
	Если элемент является директорией, то проверяется активность ключа -r: ключ активирован -- идёт работа с директорией, нет -- пропуск файла. Если элемент является файлом, то никаких проверок не проводится и начинается непосредственная работа с ним.
	
	\subsubsection*{Подготовка к сжатию}
	
	Если у имени файла есть суффикс <<.gz>>, то файл не обрабатывается. Программа по итогу выведет соответствующее сообщение. Если этого суффикса нет, то, если отсутствует ключ -с, проверяется наличие файла с тем же именем, но при этом ещё и с суффиксом. Если такой файл существует -- пользователю предлагают перезаписать этот файл. Если пользователь откажется, то работа с данным файлом прекращается.
	
	Если отсутствуют ключи -1 и -9, то файл сжимается сразу двумя методами. По итогу сжатия получаем два временных файла. Далее сравниваются размеры файлов, где выбирается наименьший, который впоследствии и становится результатом работы программы. Тот файл, который больше, просто удаляется. Если применяется один из этих ключей, то программа использует один из двух алгоритмов. Для ключа -1 это LZW, а для ключа -9 это арифметическое кодирование.
	
	В случае если какой-то из алгоритмов дал сбой, то прекращается работа с файлом и выводится соответствующая ошибка.
	
	\subsubsection*{Подготовка к распаковке}
	
	Для распаковки используется ключ -d. Проверяется наличие у файла суффикса <<.gz>>. Если он не имеется, то работа с файлом прекращается и выводится соответствующее сообщение. Если имеется, то при отсутствии -t и -c проверяется наличие файла с тем же именем, но без суффикса <<.gz>>. При наличии такого файла, программа запрашивает разрешение на перезапись. В случае отказа завершается работа с файлом. Когда такого файла нет или пользователь дал согласие, то работа продолжается. Каждый архив, сжатый этой программой имеется первые несколько служебных байт, которые содержат в себе информацию о алгоритме сжатия и размере исходного файла. При считывании первого байта определяется алгоритм сжатия. Для LZW это символ <<L>>, для арифметического кодирования это <<A>>. Если это другой символ, то работа с файлом прекращается и выводится уведомление об ошибке. В случае неудачного завершения алгоритма, выводится соответствующее сообщение. При отсутствии ключей -t и -c удаляется файл, в который записывались данные после распаковки, и работа с файлом прекращается. Далее при отсутствии ключей -t, -c и -k, удаляется изначальный архив, а при отсутствии ключей -t и -c временный файл для распаковки переименовывается и получает имя изначального архива без расширения <<.gz>>.
	
	\subsubsection*{Получение информации об архиве}
	
	В случае указания ключа -l производятся следующие действия. Для начала программа считывает первый байт. В случае, если он не совпадает с символами, указывающими на метод сжатия, то выводится сообщение об ошибке и завершается работа с файлом. 
	
	Далее читается 8 байт, в которые помещается размер файла до сжатия. После этого программа считывает размер архива, и вычисляется процент сжатия. Далее выводятся размер сжатого файла, размер до компрессии, процент сжатия в полуинтервале $[-100\%; 100\%)$ и имя файла до архивации (если файл имеет расширение <<.gz>>, то имя выводится без этого расширения, в противном случае выводится имя архива).
	
	Далее незакодированный файл будет упоминаться как файл, а закодированный файл как архив.
	
	\subsubsection*{LZW компрессия}
	
	Компрессия методом LZW происходит по следующему принципу: из размера файла определяется верхняя граница буфера, в котором будут храниться слова, и каким количеством байт будут кодироваться слова, после чего строится префиксное дерево из всех односимвольных слов-символов ASCII. В архив записывается 9 байт информации, первый из которых -- это указание метода архивации, а остальные 8 -- размер изначального файла. Далее читается первый символ файла, и в архив записывается код соответствующего слова. Полученный символ указывает на узел, в который будет добавлен следующий символ. Затем символы считываются до создания новой вершины в префиксном дереве, а в архив записывается код вершины, предшествующей новой. Последняя буква, полученная до добавления вершины заносится в буфер. После чего из корня ищется вершина, к которой ведёт эта буква, и процесс повторяется вплоть до окончания символов в файле или создания максимального количества вершин, которое сможет прочитать декомпрессор. Если произошло второе, то в архив записывается $0$, (никак иначе на этом этапе он быть записан не может), ранее установленным количеством байт, из префиксного дерева удаляются все вершины кроме корневой и потомков первого рода, после чего процесс компрессии начинается заново, но позиции в исходном файле и архиве не получают откат. Если на каком-либо этапе компрессии возникает ошибка, его работа прекращается, и выводится соответствующая ошибка.
	
	\subsubsection*{LZW декомпрессия}
	
	Декомпрессия начинается с прочтения размера изначального файла из архива, который необходим для определения нужно количества байт для прочтения слова и проверки на безошибочность декомпрессии. Далее, из архива считываются только коды слов определённого ранее размера. После считывания первого слова создаётся красно-чёрное дерево, и в него записываются все односимвольные слова из ASCII символов. Далее, по полученному коду в дереве находится необходимая строка, и она записывается в файл. После чего этот символ записывается во временное слово. Далее алгоритм считывает коды из архива. При обработке кодов возможны 4 ситуации:
	
	\begin{enumerate}
		\item Код входит в список полученных слов. В этом случае в красно-чёрном дереве ищется слово с необходимым кодом, и это слово записывается в файл, после чего в красно-чёрное дерево записывается новое слово, которое является предыдущим словом, к которому добавили первую букву только что полученного. Декомпрессия продолжается.
		\item Код не входит в красно-чёрное дерево, но он является следующим по счёту, следовательно это слово можно интерпретировать как предыдущее, к которому дописали в конце букву, с которой оно начинается. Полученное слово записывается в файл и в красно-чёрное дерево, после чего декомпрессия продолжается.
		\item Код не входит в красно-чёрное дерево и не является следующим на подходе, следовательно архив повреждён. Декомпрессия прекращается, и выводится соответствующее сообщение.
		\item Код равен $0$. Красно-чёрное дерево очищается, и процесс декомпрессии начинается сначала, но позиции в файле и архиве не получают откат.
	\end{enumerate}
	
	По окончании чтения архива, количество байт, которое было в изначальном файле, сверяется с тем, сколько было записано в его новую версию. При несовпадении выводится соответствующее сообщение, и декомпрессия завершается неудачно.
	
	\subsubsection*{Арифметическая компрессия}
	В теории арифметическое сжатие описывается достаточно просто. У нас имеется промежуток от 0 до 1. Имеется таблица частот, в которой содержится информация о том, как часто встречается тот или иной символ. Промежуток разделяется на множество отрезков, каждый из которых представляет собой какой-либо символ. При считывании символа, мы переходим к его отрезку. Далее цикл повторяется, но уже с новыми границами, заданными этим символом. На практике же мы неизбежно сталкиваемся с машинным эпсилон, поэтому следует попробовать написать всё в целых числах.
	
	На вход мы получаем файл. Создаём свой файл, в который мы заносим первые 9 байт. 1 байт обозначит тип сжатия, остальные 8 байт - размер исходного файла. По умолчанию у каждого символа частота выставлена на единицу.
	
	Каждый символ кодируется по следующей схеме:
	
	\begin{enumerate}
		\item Рассчёт границ символа по частоте его появления;
		\item Кодировка символа посредством цепочки манипуляций над границами. Если отрезок лежит в верхней половине допустимых значений - пишем бит равный единице. Если лежит в нижней половине - пишем бит равный нулю. Если лежит где-то по центру - увеличиваем счётчик битов, которые будут выставлены вслед за следующим битом с отличным от него значением. Если не выполняется ни одно из этих условий, т.е. получившийся отрезок достаточно большой, то кодировка завершается. Иначе - увеличиваем границы в 2 раза. По сути это аналогично побитовому сдвигу влево.
		\item Обновление таблицы частот. Если случилось переполнение, то масштабируем частоты, деля их на два и пересчитывая накопленные частоты. После этого производится сортировка таблицы, чтобы ускорить работу с ней.
	\end{enumerate}
	\subsubsection*{Арифметическая декомпрессия}
	В теории мы получаем число, в котором закодированы символы. Далее мы определяем в каком отрезке лежит это число, благодаря чему узнаём о закодированном символе. Далее мы выбираем новые границы, а именно границы того отрезка. Разбиваем этот отрезок также на несколько частей, следуя таблице частот и аналогично узнаём следующий символ. Однако в текущей реализации используются целые числа, поэтому и декодирование немного отличается.
	
	В первую очередь мы получаем такое же число. По нему мы также определяем границы, однако способ их нахождения несколько отличается - вместо того, чтобы сразу их узнать, мы сначала находим накопленную частоту и уже по ней определяем границы и закодированный символ. После этого мы по аналогичной схеме, как в кодировании символа, проводим манипуляции над границами и таким образом убираем ненужные биты. Далее мы начинаем декодировать следующий символ.
	
	После декодирования символа мы также обновляем таблицу частот.
	
	\subsection*{Описание файлов программы}
	
	Код программы разбит на 13 файлов:
	
	\begin{enumerate}
		\item ACC.h - Содержит перечисление методов и описание класса TACC, необходимого для работы арифметической компрессии и декомпрессии. 
		\item ACC.cpp - Содержит реализацию всех методов класса TACC.
		\item BFile.h - Содержит перечисление методов и описание класса TOutBinary и класса TInBinary, необходимых для записи в файл и чтения из файла соответственно.
		\item BFile.cpp - Содержит реализацию всех методов классов TOutBinary и TInBinary.
		\item Globals.h - Содержит в себе все необходимые глобальные переменные и библиотеки используемые несколькими файлами.
		\item LZW.h - Содержит перечисление методов и описание класса TLZW, необходимого для работы алгоритма LZW.
		\item LZW.cpp - Содержит реализацию всех методов класса TLZW.
		\item main\_help.h - Содержит в себе перечисление и описание всех функций необходимых для препроцессинга перед началом работы алгоритмов компрессии и декомпрессии.
		\item main\_help.cpp - Содержит реализацию всех функций, необходимых для препроцессинга, описанных в файле main\_help.h.
		\item Prefix.h - Содержит перечисление методов и описание класса TPrefix, необходимого для работы LZW компрессии.
		\item Prefix.cpp - Содержит реализацию всех методов класса TPrefix.
		\item main.cpp - Содержит в себе алгоритм чтения файлов и ключей.
		\item Makefile - Файл для сборки программы.
	\end{enumerate}
	
	\subsection*{Основные типы данных}
	
	\begin{enumerate}
		\item TArithmetic - класс, описывающий работу арифметического алгоритма компрессии и декомпрессии.
		\item TOutBinary - класс обеспечивающий запись необходимого количества байт в файл.
		\item TInBinary - класс обеспечивающий считывание необходимого количества байт из файла.
		\item TLZW - класс, описывающий работу алгоритма LZW.
		\item TPrefix - класс, обеспечивающий построение префиксного дерева для LZW сжатия.
	\end{enumerate}
	
	\subsection*{Описание методов и функций программы}
	
	\subsubsection*{Основные свойства и методы класса TACC}
	\noindent
	public:
	
	\begin{enumerate}
		\item bool Compress (const char*, const char*) - сжатие файла;
		\item bool Decompress (const char*, const char*) - распаковка файла;
		\item TACC() - конструктор, в котором задаются начальные значения для последующей работы со сжатием/распаковкой файла;
	\end{enumerate}
	\noindent
	private:
	
	\begin{enumerate}
		\item bool chError - флаг ошибки при распаковке файла;
		\item unsigned char indexToChar [NO\_OF\_SYMBOLS] - таблица перевода из индексов к символам;
		\item int charToIndex [NO\_OF\_CHARS] - таблица перевода из символов в индексы;
		\item int cumFreq [NO\_OF\_SYMBOLS + 1] - массив накопленных частот. Нужен для определения границ;
		\item int freq [NO\_OF\_SYMBOLS + 1] - массив частот. В нём хранится число появлений тех или иных символов;
		\item long low - нижняя граница отрезка; 
		\item long high - верхняя граница отрезка;
		\item long value - число, которое лежит в отрезке;
		\item long bitsToFollow - количество бит, которые надо пустить в след за следующим выставляемым битом;
		\item int buffer - буффер для работы с файлом;
		\item int bitsToGo - число битов, которые ещё можно загрузить в буффер;
		\item int garbageBits - счётчик плохих битов при распаковке файла. Как только их становится слишком много - распаковка отменяется и выводится сообщение об этом;
		\item FILE *out - файл, в который мы записываем;
		\item FILE *in - файл, из которого мы считываем;
		\item void UpdateModel (int) - обновление модели под новый символ;
		\item void StartInputingBits() - подготовка к побитовому вводу;
		\item void StartOutputingBits() - подготовка к побитовому выводу;
		\item void EncodeSymbol (int) - кодировка символа;
		\item void StartEncoding() - подготовка к сжатию;
		\item void DoneEncoding() - завершение кодирования. Загрузка последних битов в буффер;
		\item void StartDecoding() - подготовка к распаковке;
		\item int DecodeSymbol() - распаковка символа;
		\item int InputBit() - получение одного бита из файла;
		\item void OutputBit(int) - отправление одного бита в файл;
		\item void DoneOutputingBits() - отправление последних битов в файл;
		\item void OutputBitPlusFollow(int) - вывод указанного бита и отложенных ранее;
	\end{enumerate}
	
	\subsubsection*{Основные свойства и методы класса TOutBinary}
	\noindent
	public:
	
	\begin{enumerate}
		\item TOutBinary() - задаёт начальные значения. Файл не будет открыт.
		\item bool Open(std::string*) - открывает файл;
		\item bool Close() - закрывает файл;
		\item bool Write(const char*, size\_t) - запись в файл;
		\item bool WriteBin(size\_t bit) - запись бита в файл;
		\item unsigned long long SizeFile() - подсчёт размера файла;
		\item friend bool operator << (TOutBinary\& file, size\_t const \&bit) - запись бита в файл;
	\end{enumerate}
	\noindent
	private:
	
	\begin{enumerate}
		\item std::ofstream out - файл вывода;
		\item std::string name - имя файла;
		\item unsigned char head - маска для заноса бита в block;
		\item unsigned char block - временный буффер для хранения и записи битов в файл;
	\end{enumerate}
	
	\subsubsection*{Основные свойства и методы класса TInBinary}
	\noindent
	public:
	
	\begin{enumerate}
		\item TInBinary() - задаёт начальные значения. Файл не будет открыт.
		\item bool Open(std::string*) - открывает файл;
		\item bool Close() - закрывает файл;
		\item bool Read(char*, size\_t) - считывает из файла некоторое количество байт;
		\item bool ReadBin(char* bit) - считывает из файла один бит;
		\item unsigned long long SizeFile() - подсчёт размера файла;
		\item friend bool operator >> (TInBinary\& iFile, char \&bit) - получение бита из файла;
	\end{enumerate}
	\noindent
	private:
	
	\begin{enumerate}
		\item std::ifstream in - файл вывода;
		\item std::string name - имя файла;
		\item unsigned char head - маска для заноса бита в block;
		\item unsigned char block - временный буффер для хранения битов из файла, через него получают биты;
	\end{enumerate}
	
	\subsubsection*{Основные свойства и методы класса TLZW}
	\noindent
	public:
	
	\begin{enumerate}
		\item TLZW(TInBinary*, TOutBinary*) - Конструктор класса. Передаются файл для чтения и файл для записи.
		\item bool Compress(std::string) - Производит компрессию данных. На вход получает имя файла для компрессии. В случае успешного выполнения возвращает true, иначе false.
		\item bool Decompress(std::string) - Производит декомпрессию данных. На вход получает имя файла для декомпрессии. В случае успешного выполнения возвращает true, иначе false.
		\item $\sim$TLZW() - Стандартный деконструктор.
	\end{enumerate}
	\noindent
	private:
	
	\begin{enumerate}
		\item TInBinary* ForRead - Файл для чтения.
		\item TOutBinary* ForWrite - Файл для записи.
		\item TPrefix* CompressionTree - Префиксное дерево для хранения слов при компрессии.
		\item std::map<unsigned long long int, std::string> DecompressionTree - Красно-чёрное дерево для хранения слов при декомпрессии.
	\end{enumerate}
	
	\subsubsection*{Основные свойства и методы класса TPrefix}
	
	\noindent
	public:
	
	\begin{enumerate}
		\item TPrefix(TInBinary*, TOutBinary*) - Конструктор для корневой вершины. Передаются файл для чтения и файл для записи.
		\item TPrefix() - Конструктор для всех прочих вершин.
		\item int Update(char) - Добавление вершины из других вершин. Возвращает коды ошибок или успехов.
		\item int UpdateForRoot() - Добавление вершины из корня. Возвращает коды ошибок или успехов.
		\item void Clear(bool) - Очистка дерева после переполнения.
		\item $\sim$TPrefix() - Стандартный деконструктор.
	\end{enumerate}
	\noindent
	private:
	
	\begin{enumerate}
		\item std::vector<std::pair<char, TPrefix*>\hspace{0pt}> Next - Вектор потомков вершины и путей в них.
		\item unsigned long long int NumberOfWord - Номер слова в данном узле.
		\item static char LastLetter - Последняя прочитанная буква. Необходима для построения нового слова.
		\item static unsigned long long int NeedToRead - Вспомогательная переменная для чтения нужного кол-ва символов.
		\item static unsigned long long int LastNumber - Номер следующего добавленного слова.
		\item static unsigned long long int Border - Максимальная граница количества слов перед очисткой дерева.
		\item static TInBinary* ForRead - Файл для чтения.
		\item static TOutBinary* ForWrite - Файл для записи
		\item static unsigned short int Bites - Количество байт, необходимое для кодирования слова.
	\end{enumerate}
	
	\subsubsection*{Прочие функции}
	\noindent
	\begin{enumerate}
		\item bool KeyManager(std::string) - Обрабатывает полученные ключи. В случае получения неизвестного ключа возвращает false, иначе true.
		\item bool DifferensOfSizes(TInBinary*, std::string) - вывод для каждого файла размера сжатого, оригинального, коэффициента сжатия(\%) и имя оригинального файла(ключ l). В случае повреждения. архива возвращает false, иначе true.
		\item void WorkWithDirectory(std::string) - работает с директорией (ключ r).
		\item void WorkWithFile(std::string) - работает с файлом (определяет наличие файла, принимает решение о компрессии или декомпрессии, выполняет прочие ключи).
		\item bool IsDirectory(std::string, bool) - Проверяет, является ли файл директорией. Если файл является директорией, возвращает true, иначе false.
		\item void PrintDirectoryErrors(std::string) - Уведомляет об ошибках.
		\item bool IsArchive(std::string) - Проверяет, является ли файл архивом. Если файл является архивом, возвращает true, иначе false.
		\item void Rename(std::string, std::string) - Изменяет название файла после успешной компрессии или декомпрессии.
		\item void Delete(std::string) - Удаляет временный файл.
		\item void MainDecompress(TInBinary*, std::string) - Отвечает за подготовку декомпрессинга.
		\item void MainCompress(TInBinary*, std::string) - Отвечает за подготовку компрессинга.
		\item unsigned long long int LZWCompress(TInBinary*, std::string, TOutBinary*) - Подготавливает LZW компрессию. Возвращает размер нового файла.
		\item unsigned long long int ArithmeticCompress(TInBinary*, std::string) - Подготавливает арифметический компрессию. Возвращает размер нового файла.
		\item void KeepSmall(unsigned long long int, unsigned long long int, std::string) - Сохраняет архив самого малого размера.
		\item int main(int, char*) - Осуществляет чтение входных данных.
	\end{enumerate}
	
	\subsection*{Исходный код}%TODOпоходу его будем вставлять как всё закончим
	
	\subsection*{Тест производительности}%TODO
	
	В качестве теста производительности использовался файл размером%TODO
	, наполненный случайными символами английского алфавита.
	
	\noindent
	\begin{tabular}{| l | l | l | l | l | l |}
		\hline
		& Размер     & Эффективность & Максимальное & Время      & Время  \\
		& сжатого    & сжатия        & потребление  & компрессии & декомпрессии  \\
		& файла (б) &               & памяти (Мб)  & (с)        & (с) \\
		\hline
		gzip        & & & & & \\
		\hline
		Оба         & & & & & \\
		алгоритма   & & & & & \\
		\hline
		LZW         & & & & & \\
		\hline
		Арифм.      & & & & & \\
		кодирование & & & & & \\
		\hline
	\end{tabular}
	
	%TODO ВВеди характеристики компа
	
	\subsection*{Выводы}
	
	В процессе выполнения данной работы я получил некоторые знания и навыки связанные с компрессией и декомпрессией файлов. Так же я закрепил полученные ранее знания о работе с префиксными деревьями. Были освоены новые приёмы работы с файлами и получены базовые навыки для работ с директориями. 
	
	LZW кодирование получилось понять и реализовать за гораздо меньшее время и написав меньше строчек кода чем с арифметическим кодированием, но при этом арифметическое кодирование сжимает данные гораздо лучше.%TODO почему
	
	\subsection*{Список литературы}
	\begin{enumerate}
		\item Арифметическое кодирование [Электронный ресурс]: mf.grsu.by URL:\\ http://mf.grsu.by/UchProc/livak/po/comprsite/theory\_arithmetic.html (дата обращения 14.07.2020)
		\item Арифметическое кодирование [Электронный ресурс]: habr.com URL:\\ https://habr.com/ru/post/130531/ (дата обращения 26.08.2020)
		\item Алгоритм LZW [Электронный ресурс]: mf.grsu.by URL:\\ http://mf.grsu.by/UchProc/livak/po/comprsite/theory\_lzw.html \\(дата обращения 26.08.2020)
	\end{enumerate}
\end{document}